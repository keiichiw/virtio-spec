\section{Video Device}\label{sec:Device Types / Video Device}

The virtio video encoder and decoder devices are virtual devices that support
video encoding and decoding respectively. Despite being different devices, they
use the same protocol.

\subsection{Device ID}
\label{sec:Device Types / Video Device / Device ID}

\begin{description}
\item[30] encoder device
\item[31] decoder device
\end{description}

\subsection{Virtqueues}
\label{sec:Device Types / Video Device / Virtqueues}

\begin{description}
\item[0] commandq - queue for driver commands and device responses to these
  commands.
\item[1] eventq - queue for events sent by the device to the driver.
\end{description}

\subsection{Feature bits}
\label{sec:Device Types / Video Device / Feature bits}

\begin{description}
\item[VIRTIO_VIDEO_F_RESOURCE_GUEST_PAGES (0)] Guest pages can be used for video
  buffers.
\item[VIRTIO_VIDEO_F_RESOURCE_NON_CONTIG (1)] The device can use
  non-contiguous memory for video buffers. Without this flag, the
  driver and device MUST use video buffers that are contiguous for the device.
\item[VIRTIO_VIDEO_F_RESOURCE_VIRTIO_OBJECT (2)] Objects exported by
  another virtio device can be used as video buffers.
\end{description}

\devicenormative{\subsubsection}{Feature bits}{Device Types / Video
  Device / Feature bits}

The device MUST present at least one of VIRTIO_VIDEO_F_RESOURCE_GUEST_PAGES or
VIRTIO_VIDEO_F_RESOURCE_VIRTIO_OBJECT.

\subsection{Device configuration layout}
\label{sec:Device Types / Video Device / Device configuration layout}

Video device configuration uses the following layout structure:

\begin{lstlisting}
struct virtio_video_config {
        le32 version;
        le32 caps_length;
};
\end{lstlisting}

\begin{description}
\item[\field{version}] is the protocol version that the device understands. The
  device MUST set this to 0.
\item[\field{caps_length}] is the length of a device-writable descriptor
  required to call VIRTIO_VIDEO_CMD_DEVICE_QUERY_CAPS in bytes. The device MUST
  set this value.
\end{description}

\subsection{Supported formats}

\subsubsection{Supported image formats}

The following image formats are defined:
\begin{lstlisting}
enum virtio_video_image_format {
        /* Raw formats */
        VIRTIO_VIDEO_IMAGE_FORMAT_ARGB8888 = 1,
        VIRTIO_VIDEO_IMAGE_FORMAT_BGRA8888,
        VIRTIO_VIDEO_IMAGE_FORMAT_NV12,   /* 12  Y/CbCr 4:2:0  */
        VIRTIO_VIDEO_IMAGE_FORMAT_YUV420, /* 12  YUV 4:2:0     */
        VIRTIO_VIDEO_IMAGE_FORMAT_YVU420, /* 12  YVU 4:2:0     */
};
\end{lstlisting}

\subsubsection{Supported bitstream formats}

The following bitstream formats are defined:
\begin{lstlisting}
enum virtio_video_codec {
        VIRTIO_VIDEO_CODEC_MPEG2 = 1, /* MPEG-2 Part 2 */
        VIRTIO_VIDEO_CODEC_MPEG4,     /* MPEG-4 Part 2 */
        VIRTIO_VIDEO_CODEC_H264,      /* H.264 */
        VIRTIO_VIDEO_CODEC_HEVC,      /* HEVC aka H.265*/
        VIRTIO_VIDEO_CODEC_VP8,       /* VP8 */
        VIRTIO_VIDEO_CODEC_VP9,       /* VP9 */
};

union virtio_video_codec_params {
        struct virtio_video_codec_h264 h264;
        struct virtio_video_codec_hevc hevc;
        struct virtio_video_codec_vp8 vp8;
        struct virtio_video_codec_vp9 vp9;
}

struct virtio_video_bitstream_format {
        le32 virtio_video_codec codec; /* VIRTIO_VIDEO_CODEC_* */
        union virtio_video_codec_params params;
};
\end{lstlisting}

The field \field{params} in \field{struct virtio_video_bitstream_format} is
valid only when \field{codec} designates a value which has a corresponding field
in \field{union virtio_video_codec_params}.

The fields in \field{union virtio_video_codec_params} are defined below.

\paragraph{H.264}

When the field \field{codec} in \field{struct virtio_video_bitstream_format}
is set to VIRTIO_VIDEO_CODEC_H264, \field{params.h264} MUST be set to a
valid value defined as follows:

\begin{lstlisting}
enum virtio_video_codec_h264_profile {
        VIRTIO_VIDEO_CODEC_H264_PROFILE_BASELINE = 1,
        VIRTIO_VIDEO_CODEC_H264_PROFILE_MAIN,
        VIRTIO_VIDEO_CODEC_H264_PROFILE_EXTENDED,
        VIRTIO_VIDEO_CODEC_H264_PROFILE_HIGH,
        VIRTIO_VIDEO_CODEC_H264_PROFILE_HIGH10PROFILE,
        VIRTIO_VIDEO_CODEC_H264_PROFILE_HIGH422PROFILE,
        VIRTIO_VIDEO_CODEC_H264_PROFILE_HIGH444PREDICTIVEPROFILE,
        VIRTIO_VIDEO_CODEC_H264_PROFILE_SCALABLEBASELINE,
        VIRTIO_VIDEO_CODEC_H264_PROFILE_SCALABLEHIGH,
        VIRTIO_VIDEO_CODEC_H264_PROFILE_STEREOHIGH,
        VIRTIO_VIDEO_CODEC_H264_PROFILE_MULTIVIEWHIGH,
};

enum virtio_video_codec_h264_level {
        VIRTIO_VIDEO_CODEC_H264_LEVEL_1_0 = 1,
        VIRTIO_VIDEO_CODEC_H264_LEVEL_1_1,
        VIRTIO_VIDEO_CODEC_H264_LEVEL_1_2,
        VIRTIO_VIDEO_CODEC_H264_LEVEL_1_3,
        VIRTIO_VIDEO_CODEC_H264_LEVEL_2_0,
        VIRTIO_VIDEO_CODEC_H264_LEVEL_2_1,
        VIRTIO_VIDEO_CODEC_H264_LEVEL_2_2,
        VIRTIO_VIDEO_CODEC_H264_LEVEL_3_0,
        VIRTIO_VIDEO_CODEC_H264_LEVEL_3_1,
        VIRTIO_VIDEO_CODEC_H264_LEVEL_3_2,
        VIRTIO_VIDEO_CODEC_H264_LEVEL_4_0,
        VIRTIO_VIDEO_CODEC_H264_LEVEL_4_1,
        VIRTIO_VIDEO_CODEC_H264_LEVEL_4_2,
        VIRTIO_VIDEO_CODEC_H264_LEVEL_5_0,
        VIRTIO_VIDEO_CODEC_H264_LEVEL_5_1,
};

struct virtio_video_codec_h264 {
        le32 profile; /* VIRTIO_VIDEO_CODEC_H264_PROFILE_* */
        le32 level;   /* VIRTIO_VIDEO_CODEC_H264_LEVEL_* */
};
\end{lstlisting}

\paragraph{HEVC}

When the field \field{codec} in \field{struct virtio_video_bitstream_format}
is set to VIRTIO_VIDEO_CODEC_HEVC, \field{params.hevc} MUST be set to a
valid value defined as follows:

\begin{lstlisting}
enum virtio_video_codec_hevc_profile {
        VIRTIO_VIDEO_CODEC_HEVC_PROFILE_MAIN = 1,
        VIRTIO_VIDEO_CODEC_HEVC_PROFILE_MAIN10,
        VIRTIO_VIDEO_CODEC_HEVC_PROFILE_MAIN_STILL_PICTURE,
};

struct virtio_video_codec_hevc {
        le32 profile; /* VIRTIO_VIDEO_CODEC_HEVC_PROFILE_* */
};
\end{lstlisting}

\paragraph{VP8}

When the field \field{codec} in \field{struct virtio_video_bitstream_format}
is set to VIRTIO_VIDEO_CODEC_VP8, \field{params.vp8} MUST be set to a
valid value defined as follows:

\begin{lstlisting}
enum virtio_video_codec_vp8_profile {
        VIRTIO_VIDEO_CODEC_VP8_PROFILE_0 = 1,
        VIRTIO_VIDEO_CODEC_VP8_PROFILE_1,
        VIRTIO_VIDEO_CODEC_VP8_PROFILE_2,
        VIRTIO_VIDEO_CODEC_VP8_PROFILE_3,
};

struct virtio_video_codec_vp8 {
        le32 profile;/* VIRTIO_VIDEO_CODEC_VP8_PROFILE_* */
};
\end{lstlisting}

\paragraph{VP9}

When the field \field{codec} in \field{struct virtio_video_bitstream_format}
is set to VIRTIO_VIDEO_CODEC_VP9, \field{params.vp9} MUST be set to a
valid value defined as follows:

\begin{lstlisting}
enum virtio_video_codec_vp9_profile {
        VIRTIO_VIDEO_CODEC_VP9_PROFILE_0 = 1,
        VIRTIO_VIDEO_CODEC_VP9_PROFILE_1,
        VIRTIO_VIDEO_CODEC_VP9_PROFILE_2,
        VIRTIO_VIDEO_CODEC_VP9_PROFILE_3,
};

struct virtio_video_codec_vp9 {
        le32 profile;/* VIRTIO_VIDEO_CODEC_VP9_PROFILE_* */
};
\end{lstlisting}

\subsection{Device Initialization}
\label{sec:Device Types / Video Device / Device Initialization}

\devicenormative{\subsubsection}{Device Initialization}{Device Types /
  Video Device / Device Initialization}

The driver SHOULD query device capability by using the
VIRTIO_VIDEO_CMD_DEVICE_QUERY_CAPS and use that information for the
initial setup.

\subsection{Device Operation}
\label{sec:Device Types / Video Device / Device Operation}

After initializing the device and setting the desired parameters,
the driver allocates input and output buffers and queues them
to the device. The device then performs the requested operation on the buffers.

\subsubsection{Command Virtqueue}

The command virtqueue is used to send commands and their responses. Commands
MUST be written by the driver and their responses MUST be written by the device
in the next device-writable descriptor.

Different structure layouts are used for each command and response. Every
command structure starts with a field storing a variant of \field{enum
  virtio_video_cmd_type} defined as follows:
\begin{lstlisting}
enum virtio_video_cmd_type {
        /* Global */
        VIRTIO_VIDEO_CMD_DEVICE_QUERY_CAPS = 0x0100,

        /* Stream */
        VIRTIO_VIDEO_CMD_STREAM_CREATE = 0x0200,
        VIRTIO_VIDEO_CMD_STREAM_DESTROY,
        VIRTIO_VIDEO_CMD_STREAM_GET_PARAMS,
        VIRTIO_VIDEO_CMD_STREAM_SET_PARAMS,
        VIRTIO_VIDEO_CMD_STREAM_DRAIN,

        /* Queue */
        VIRTIO_VIDEO_CMD_QUEUE_CLEAR = 0x300,
        VIRTIO_VIDEO_CMD_QUEUE_DETACH_RESOURCES,

        /* Resource*/
        VIRTIO_VIDEO_CMD_RESOURCE_ATTACH = 0x400,
        VIRTIO_VIDEO_CMD_RESOURCE_QUEUE,
};
\end{lstlisting}

Each response structure starts with a field storing a variant of
\field{enum virtio_video_result} defined as follows:
\begin{lstlisting}
enum virtio_video_result {
         /* Success */
         VIRTIO_VIDEO_RESULT_OK = 0x000,

         /* Error */
         VIRTIO_VIDEO_RESULT_ERR_INVALID_OPERATION = 0x0100,
         VIRTIO_VIDEO_RESULT_ERR_INVALID_STREAM_ID,
         VIRTIO_VIDEO_RESULT_ERR_INVALID_RESOURCE_ID,
         VIRTIO_VIDEO_RESULT_ERR_INVALID_PARAMETER,
         VIRTIO_VIDEO_RESULT_ERR_CANCELED,
         VIRTIO_VIDEO_RESULT_ERR_OUT_OF_MEMORY,
};
\end{lstlisting}

\paragraph{Query device capability}

\begin{description}
\item[VIRTIO_VIDEO_CMD_DEVICE_QUERY_CAPS] Retrieve information about device
  capabilities.

The driver sends this command with \field{struct
  virtio_video_device_query_caps}.
\begin{lstlisting}
struct virtio_video_device_query_caps {
        le32 cmd_type;
};
\end{lstlisting}
\begin{description}
\item[\field{cmd_type}] MUST be set to VIRTIO_VIDEO_CMD_DEVICE_QUERY_CAPS.
\end{description}

The device responds with \field{struct virtio_video_device_query_caps_resp}.
\begin{lstlisting}
struct virtio_video_device_query_caps_resp {
        le32 result; /* VIRTIO_VIDEO_RESULT_* */
        le32 num_image_formats;
        le32 num_bitstream_formats;
        /**
         * Followed by
         * struct virtio_video_image_format_desc image_formats[num_image_formats];
         */
        /**
         * Followed by
         * struct virtio_video_bitstream_format_desc bitstream_formats[num_bitstream_formats];
         */
};
\end{lstlisting}
\begin{description}
\item[\field{result}] MUST be set to one of variants of \field{enum
    virtio_video_result}.
\item[\field{num_image_formats}] is the number of supported image formats. If
  \field{result} is not VIRTIO_VIDEO_RESULT_OK, the device MUST set this to 0.
\item[\field{num_bitstream_formats}] is the number of supported bitstream
  formats. If \field{result} is not VIRTIO_VIDEO_RESULT_OK, the device MUST set
  this to 0.
\end{description}

The device MUST write two arrays of \field{struct
  virtio_video_image_format_desc} and \field{struct
  virtio_video_image_format_desc} following the \field{struct
  virtio_video_device_query_caps_resp}. The lengths of these arrays MUST be
\field{num_image_formats} and \field{num_bitstream_formats} respectively.
\subparagraph{Image format description}

The image format description \field{virtio_video_image_format_desc} is defined
as follows:
\begin{lstlisting}
enum virtio_video_planes_layout {
        VIRTIO_VIDEO_PLANES_LAYOUT_SINGLE_BUFFER = 0,
        VIRTIO_VIDEO_PLANES_LAYOUT_PER_PLANE,
};

struct virtio_video_image_format_desc {
         le32 format; /* VIRTIO_VIDEO_CMD_IMAGE_FORMAT_* */
         le32 planes_layouts; /* Bitmask with VIRTIO_VIDEO_PLANES_LAYOUT_* */
         le32 plane_align;
         le32 num_frames;
         /**
          * Followed by
          * struct virtio_video_image_format_frame frames[num_frames];
          */
};
\end{lstlisting}
\begin{description}
\item[\field{format}] specifies an image format. The device MUST set
  one of the variants of \field{enum virtio_video_image_format}.
\item[\field{planes_layouts}] is a bitmask representing the set of plane layout
  types that the device supports.
  \begin{description}
  \item[\field{VIRTIO_VIDEO_PLANES_LAYOUT_SINGLE_BUFFER}] The device expects
    the planes of a frame to be laid out one after another in the same buffer.
  \item[\field{VIRTIO_VIDEO_PLANES_LAYOUT_PER_PLANE}] The device expects the
    planes of a frame to be located in separate buffers.
  \end{description}
\item[\field{plane_align}] is the plane alignment the device requires when
  multiple planes are located in the same buffer. This field is valid only if
  \field{planes_layouts} has the \field{VIRTIO_VIDEO_PLANES_LAYOUT_SINGLE_BUFFER}
  bit, and MUST be set to zero otherwise.
\item[\field{num_frames}] is the number of \field{virtio_video_format_frame}
  that follow.
\end{description}

The frame information \field{virtio_video_format_frame} is defined as follows:
\begin{lstlisting}
struct virtio_video_image_format_range {
       le32 min;
       le32 max;
       le32 step;
};

struct virtio_video_image_format_frame {
       struct virtio_video_image_format_range width;
       struct virtio_video_image_format_range height;
       le32 num_rates;
       /**
        * Followed by
        * struct virtio_video_image_format_range frame_rates[num_rates];
        */
};
\end{lstlisting}

The value of \field{struct virtio_video_image_format_range} is used to represent
the range of values supported by the device. The device MUST set \field{step} to
a positive integer. An integer \(x\) is in a range \field{struct
  virtio_video_image_format_range r} if \(\field{r.min} \le x \le
\field{r.max}\) holds and \(x\) is equals to \((\field{min} + \field{step} *
n)\) for some integer \(n\).

\begin{description}
\item[\field{width}] represents the range of widths supported by the device.
\item[\field{height}] represents the range of heights supported by the device.
\item[\field{num_rates}] is the length of the \field{frame_rates[]} array.
\item[\field{frame_rates}] is the set of frame rates ranges supported
  by the device.
\end{description}

\subparagraph{Bitstream format description}

The bitstream format description \field{virtio_video_bitstream_format_desc} is
defined as follows:
\begin{lstlisting}
struct virtio_video_bitstream_format_desc {
        le32 virtio_video_codec codec;
        le32 num_variants;
        /**
         * Followed by
         * union virtio_video_codec_params variants[num_variants];
         */
};
\end{lstlisting}
\begin{description}
\item[\field{codec}] specifies a codec. This field MUST be set to one of the
  variants of \field{enum virtio_video_codec}.
\item[\field{num_variants}] is the number of \field{union virtio_video_codec_params}
  values that follow. If \field{union virtio_video_codec_params} has a field
  corresponding to the specified \field{codec}, this value MUST be a positive
  integer. Otherwise, it MUST be 0.
\item[\field{variants}] is a list of supported variants of the given
  \field{codec}. Only the field corresponding to the given \field{codec} MUST be
  valid in each element.
\end{description}
\end{description}

\devicenormative{\subparagraph}{Query device capability}{Device Types / Video
  Device / Device Operation / Device Operation: Query device capability}

The total size of the device response MUST be equals to \field{caps_length}
bytes, as reported by the device configuration.

\paragraph{Per stream operations}

\begin{description}
\item[VIRTIO_VIDEO_CMD_STREAM_CREATE] Create a video stream using the device.

The driver sends this command with \field{struct virtio_video_stream_create}.
\begin{lstlisting}
enum virtio_video_mem_type {
       VIRTIO_VIDEO_MEM_TYPE_GUEST_PAGES = 1,
       VIRTIO_VIDEO_MEM_TYPE_VIRTIO_OBJECT,
};

struct virtio_video_stream_create {
        le32 cmd_type;
        le32 in_mem_type; /* One of VIRTIO_VIDEO_MEM_TYPE_* types */
        le32 out_mem_type; /* One of VIRTIO_VIDEO_MEM_TYPE_* types */
        le32 codec; /* One of VIRTIO_VIDEO_CODEC_* types */
};
\end{lstlisting}
\begin{description}
\item[\field{cmd_type}] MUST be set to VIRTIO_VIDEO_CMD_STREAM_CREATE.
\item[\field{in_mem_type, out_mem_type}] is the type of buffer management for
  input/output buffers. The driver MUST set a value in \field{enum
    virtio_video_mem_type} for which the device reported a corresponding feature
  bit.
\begin{description}
\item[\field{VIRTIO_VIDEO_MEM_TYPE_GUEST_PAGES}] Use guest pages. The driver
  MUST not set this value if the feature bit VIRTIO_VIDEO_F_RESOURCE_GUEST_PAGES
  is not set.
\item[\field{VIRTIO_VIDEO_MEM_TYPE_VIRTIO_OBJECT}] Use object exported by
  another virtio device. The driver MUST not set this value if the feature bit
  VIRTIO_VIDEO_F_RESOURCE_VIRTIO_OBJECT is not set.
\end{description}
\item[\field{codec}] is the video codec that will be used with this stream. The
  driver MUST set it to one of the variants of \field{enum virtio_video_codec}.
\end{description}

The device responds with \field{struct virtio_video_stream_create_resp}:
\begin{lstlisting}
struct virtio_video_stream_create_resp {
        le32 result; /* VIRTIO_VIDEO_RESULT_* */
        le32 stream_id;
};
\end{lstlisting}
\begin{description}
\item[\field{result}] MUST be set to one of the variants of \field{enum
    virtio_video_result}.
\item[\field{stream_id}] is the ID of the created stream, allocated by the
  device. It is only valid if \field{result} is VIRTIO_VIDEO_RESULT_OK.
\end{description}

\item[VIRTIO_VIDEO_CMD_STREAM_DESTROY] Destroy a video stream.

The driver sends this command with \field{struct virtio_video_stream_destroy}.
\begin{lstlisting}
struct virtio_video_stream_destroy {
         le32 cmd_type;
         le32 stream_id;
};
\end{lstlisting}
\begin{description}
\item{\field{cmd_type}} MUST be set to VIRTIO_VIDEO_CMD_STREAM_DESTROY.
\item{\field{stream_id}} is the ID of the stream to be destroyed. It must be set
to the value of an existing stream.
\end{description}

The device responds with \field{struct virtio_video_stream_destroy_resp}.
\begin{lstlisting}
struct virtio_video_stream_destroy_resp {
         le32 result; /* VIRTIO_VIDEO_RESULT_* */
};
\end{lstlisting}
\begin{description}
\item[\field{result}] MUST be set to one of the variants of \field{enum
    virtio_video_result}.
\end{description}

\item[VIRTIO_VIDEO_CMD_STREAM_GET_PARAMS] Get the current stream parameters.

The driver sends this command with \field{struct
  virtio_video_stream_get_params}.
\begin{lstlisting}
struct virtio_video_stream_get_params {
    le32 cmd_type;
    le32 stream_id;
};
\end{lstlisting}
\begin{description}
\item[\field{cmd_type}] MUST be set to VIRTIO_VIDEO_CMD_STREAM_GET_PARAMS.
\item[\field{stream_id}] MUST be set to the ID of an existing stream, whose
  parameters will be returned.
\end{description}

The device responds with \field{struct virtio_video_stream_get_params_resp}.
\begin{lstlisting}
struct virtio_video_stream_get_params_resp {
    le32 result; /* VIRTIO_VIDEO_RESULT_* */
    struct virtio_video_params params;
};
\end{lstlisting}
\begin{description}
\item[\field{result}] MUST be set to one of variants of \field{enum
    virtio_video_result}.
\item[\field{params}] is the stream parameters. It is only valid if
  \field{result} is VIRTIO_VIDEO_RESULT_OK.
\end{description}

The struct \field{virtio_video_params} is defined as follows. Some of these
parameters can be updated by the driver with VIRTIO_VIDEO_CMD_SET_PARAMS.
\begin{lstlisting}
struct virtio_video_crop {
        le32 left;
        le32 top;
        le32 width;
        le32 height;
};

enum virtio_video_rate_control {
        VIRTIO_VIDEO_RATE_CONTROL_FRAME = 1,
        VIRTIO_VIDEO_RATE_CONTROL_MACRO_BLOCK,
};

struct virtio_video_rc_range {
        le32 min_qp;
        le32 max_qp;
};

struct virtio_video_frame_qp {
        le32 iframe_qp;
        le32 pframe_qp;
        le32 bframe_qp;
};

union virtio_video_quantization_param {
        struct virtio_video_rc_range rc;
        struct virtio_video_frame_qp no_rc;
};

struct virtio_video_params {
        /* Image format */
        le32 image_format; /* VIRTIO_VIDEO_IMAGE_FORMAT_* */
        le32 min_image_buffers;
        le32 max_image_buffers;
        le32 cur_image_buffers;
        le32 width;
        le32 height;
        struct virtio_video_crop crop;
        le32 frame_rate;
        le32 planes_layout;
        le32 num_planes;
        struct virtio_video_plane_format
        plane_formats[VIRTIO_VIDEO_MAX_PLANES];

        /* Bitstream format */
        struct virtio_video_bitstream_format bitstream_format;
        le32 min_bitstream_buffers;
        le32 max_bitstream_buffers;
        le32 cur_bitstream_buffers;

        /* Bitrate  (for encoder) */
        le32 min_bitrate;
        le32 max_bitrate;
        le32 cur_bitrate;

        /* Quantization parameter (for encoder) */
        le32 rc_mode; /* VIRTIO_VIDEO_RATE_CONTROL_* */
        union virtio_video_quantization_param qp;
};
\end{lstlisting}

\begin{description}
\item[\field{image_format}] is the image format used by the stream. It is
  set to one of the variants of \field{enum virtio_video_image_format}.
\item[\field{min_image_buffers}] is the minimum number of image buffers that the
  device requires. The device MUST set this to a non-zero integer.
\item[\field{max_image_buffers}] is the maximum number of image buffers that the
  device can accept. The device MUST set this to an integer larger than
  or equal to \field{min_image_buffers}.
\item[\field{cur_image_buffers}] is the number of image buffers that the driver
  can enqueue via VIRTIO_VIDEO_RESOURCE_QUEUE. These fields are set by the
  driver to how many buffers it wishes to use, and set by the device
  to how many buffers it allows to use.
  The value MUST be larger than or equal to \field{min_image_buffers} and not
  exceed \field{max_image_buffers}.
\item[\field{width}] is the current width of frames in the stream.
\item[\field{height}] is the current height of frames in the stream.
\item[\field{crop}] is the current cropping rectangle for frames in the stream.
\item[\field{frame_rate}] is the frame rate.
\item[\field{planes_layout}] specifies the plane layout of the resource. The
  driver MUST set this to one of the variants of \field{enum
    virtio_video_planes_layout} that is supported for a current image format.
\item[\field{num_planes}] is the number of planes per frame.
\item[\field{plane_formats}] is an array containing the current planes format.
  Only the first |\field{num_planes}| elements are valid. The struct
  \field{virtio_video_plane_format} is defined as follows.
\begin{lstlisting}
struct virtio_video_plane_format {
        le32 plane_size;
        le32 stride;
        le32 offset;
};
\end{lstlisting}
\begin{description}
\item[plane_size] is a size of the plane in bytes.
\item[stride] is the line stride.
\item[offset] is an offset from the beginning of the buffer. This field is valid
  only if VIRTIO_VIDEO_PLANES_LAYOUT_SINGLE_BUFFER is set to
  \field{planes_layout}.
\end{description}
\item[\field{bitstream_format}] is the bitstream format used in the
  stream.
\item[\field{min_bitstream_buffers}] is the minimum number of bitstream buffers
  that the device requires. The device MUST set this to a non-zero integer.
\item[\field{max_bitstream_buffers}] is the maximum number of bitstream buffers
  that the device can accept. The device MUST set this to an integer which is
  larger than or equal to \field{min_bitstream_buffers}.
\item[\field{cur_bitstream_buffers}] is the number of bitstream buffers that the
  driver can enqueue via VIRTIO_VIDEO_RESOURCE_QUEUE. These fields are set by
  the driver to how many buffers it wishes to use, and set by the device to how
  many buffers it allows to use. The value MUST be larger than or equal to
  \field{min_bitstream_buffers} and not exceed \field{max_bitstream_buffers}.
\item[\field{min_bitrate}] is the minimum bitrate supported by the
  device. (only for encoders)
\item[\field{max_bitrate}] is the maximum bitrate supported by the
  device. (only for encoders)
\item[\field{cur_bitrate}] is the current bitrate of the stream. (only for
  encoders)
\item[\field{rc_mode}] whether rate control is enabled for the stream. If
  this value is one of variants of \field{enum virtio_video_rate_control}, the
  rate control described below is enabled. Otherwise, the quantization parameter
  for each frame type is constant and set with \field{qp.no_rc}. (only for
  encoders)
  \begin{description}
  \item[\field{VIRTIO_VIDEO_RATE_CONTROL_FRAME}] Frame level rate control is
    enabled. In this mode, the quantization parameter is adjusted according to
    \field{cur_bitrate}. Minimum and maximum value for the quantization
    parameter can be set with \field{qp.rc}.
  \item[\field{VIRTIO_VIDEO_RATE_CONTROL_MACRO_BLOCK}] Macroblock level rate
    control is enabled. (only for MPEG4 and H.264 encoders)
  \end{description}
\item[\field{qp}] is the quantization parameter. (only for encoders)
\end{description}

After the device responds to VIRTIO_VIDEO_CMD_STREAM_GET_PARAMS, it MUST keep
responding the same value until one of the followings events occurs:
\begin{itemize}
\item the driver sends VIRTIO_VIDEO_CMD_STREAM_SET_PARAMS via commandq.
\item the device sends VIRTIO_VIDEO_EVENT_DECODER_RESOLUTION_CHANGED via eventq.
\end{itemize}

\item[VIRTIO_VIDEO_CMD_STREAM_SET_PARAMS] Update the current stream parameters.

The driver sends this command with \field{struct
  virtio_video_stream_set_params}.
\begin{lstlisting}
struct virtio_video_stream_set_params {
        le32 cmd_type;
        le32 stream_id;
        struct virtio_video_params params;
};
\end{lstlisting}
\begin{description}
\item[\field{cmd_type}] MUST be VIRTIO_VIDEO_CMD_TYPE_SET_PARAMS.
\item[\field{stream_id}] is the ID of the stream whose parameters are to be
  updated.
\item[\field{params}] the new parameters requested by the driver.
\end{description}

The device responds with \field{struct virtio_video_stream_set_params_resp}.
\begin{lstlisting}
struct virtio_video_stream_set_params_resp {
       le32 result; /* VIRTIO_VIDEO_RESULT_* */
       struct virtio_video_params params;
};
\end{lstlisting}
\begin{description}
\item[\field{result}] MUST be set to one of the variants of \field{enum
    virtio_video_result}.
\item[\field{params}] the updated stream parameters. Values may differ from the
    requested one depending on the device's and codec capabilities, and values
    that are not directly changed by the SET_PARAMS command may also be changed.
    It is the responsibility of the driver to check all values for potential
    changes and update its state accordingly.
\end{description}

\item[VIRTIO_VIDEO_CMD_STREAM_DRAIN] Complete processing all queued input
  buffers.

VIRTIO_VIDEO_CMD_STREAM_DRAIN ensures that all sent
VIRTIO_VIDEO_CMD_RESOURCE_QUEUE commands for input buffers have been processed
by the device and that related output buffers are available to the driver.

The driver sends this command with \field{struct virtio_video_stream_drain}.
\begin{lstlisting}
struct virtio_video_stream_drain {
        le32 cmd_type;
        le32 stream_id;
};
\end{lstlisting}
\begin{description}
\item[\field{cmd_type}] MUST be VIRTIO_VIDEO_CMD_TYPE_STREAM_DRAIN.
\item[\field{stream_id}] is a valid stream ID.
\end{description}

The device responds with \field{struct virtio_video_stream_drain_resp}.
\begin{lstlisting}
struct virtio_video_stream_drain_resp {
        le32 result; /* VIRTIO_VIDEO_RESULT_* */
};
\end{lstlisting}
\begin{description}
\item[\field{result}] MUST be set to one of the variants of \field{enum
    virtio_video_result}.
\end{description}

\begin{itemize*}
\item Before the device sends the response of a VIRTIO_VIDEO_CMD_STREAM_DRAIN
  command, it MUST process and respond to all of
  VIRTIO_VIDEO_CMD_RESOURCE_* commands for the input queue which are sent before
  the drain command.
\item While the device is processing a VIRTIO_VIDEO_CMD_STREAM_DRAIN
  command, it MUST return
  VIRTIO_VIDEO_RESP_ERR_INVALID_OPERATION to the following incoming commands:
  \begin{itemize*}
  \item VIRTIO_VIDEO_CMD_RESOURCE_* commands with an input buffer, or
  \item VIRTIO_VIDEO_CMD_STREAM_DRAIN commands.
  \end{itemize*}
\item If the processing was stopped due to
  VIRTIO_VIDEO_CMD_QUEUE_CLEAR, the device MUST respond with
  VIRTIO_VIDEO_RESP_OK_NODATA as response type and
  VIRTIO_VIDEO_BUFFER_FLAG_ERR in \field{flags}.
\end{itemize*}
\end{description}

\paragraph{Per queue commands}

\begin{description}
\item[VIRTIO_VIDEO_CMD_QUEUE_CLEAR] Discard all pending resource commands for a
  given queue.

The driver sends this command with \field{struct virtio_video_queue_clear}.
\begin{lstlisting}
enum virtio_video_queue_type {
        VIRTIO_VIDEO_QUEUE_TYPE_INPUT = 1,
        VIRTIO_VIDEO_QUEUE_TYPE_OUTPUT,
};

struct virtio_video_queue_clear {
        le32 cmd_type;
        le32 stream_id;
        le32 queue_type; /* One of VIRTIO_VIDEO_QUEUE_TYPE_* types */
};
\end{lstlisting}
\begin{description}
\item[\field{cmd_type}] MUST be VIRTIO_VIDEO_CMD_TYPE_QUEUE_CLEAR.
\item[\field{stream_id}] is a valid stream ID.
\item[\field{queue_type}] is the queue to be cleared. This MUST be one of
  variants of \field{enum virtio_video_queue_type}.
\end{description}

The device responds with \field{struct virtio_video_queue_clear_resp}.
\begin{lstlisting}
struct virtio_video_queue_clear_resp {
        le32 result; /* VIRTIO_VIDEO_RESULT_* */
};
\end{lstlisting}
\begin{description}
\item[\field{result}] MUST be set to one of the variants of \field{enum
    virtio_video_result}.
\end{description}

\begin{itemize*}
\item Before the device sends a response for VIRTIO_VIDEO_CMD_QUEUE_CLEAR, it
  MUST respond with VIRTIO_VIDEO_RESULT_ERR_CANCELED to the following pending
  commands:
  \begin{itemize*}
  \item VIRTIO_VIDEO_CMD_RESOURCE_* on the input queue,
  \item VIRTIO_VIDEO_CMD_STREAM_DRAIN.
  \end{itemize*}
\item While the device is processing a VIRTIO_VIDEO_CMD_QUEUE_CLEAR, it
  MUST return VIRTIO_VIDEO_RESP_ERR_INVALID_OPERATION to the following incoming
  commands:
  \begin{itemize*}
  \item VIRTIO_VIDEO_CMD_STREAM_DRAIN,
  \item VIRTIO_VIDEO_CMD_QUEUE_CLEAR, or
  \item VIRTIO_VIDEO_CMD_RESOURCE_*.
  \end{itemize*}
\end{itemize*}

\item[VIRTIO_VIDEO_CMD_QUEUE_DETACH_RESOURCES] Detach all the resources of an
  input or an output queue.

The command VIRTIO_VIDEO_CMD_QUEUE_DETACH_RESOURCES is used to detach all
resources attached by VIRTIO_VIDEO_CMD_RESOURCE_ATTACH for a given queue.
The driver sends this command with \field{struct
  virtio_video_queue_detach_resources}.
\begin{lstlisting}
struct virtio_video_queue_detach_resources {
        le32 cmd_type;
        le32 stream_id;
        le32 queue_type; /* One of VIRTIO_VIDEO_QUEUE_TYPE_* types */
};
\end{lstlisting}
\begin{description}
\item[\field{cmd_type}] MUST be VIRTIO_VIDEO_CMD_QUEUE_DETACH_RESOURCES.
\item[\field{stream_id}] is a stream ID.
\item[\field{queue_type}] MUST be a variant of \field{enum
    virtio_video_queue_type}.
\end{description}

The device responds with \field{virtio_video_queue_detach_resources_resp}.
\begin{lstlisting}
struct virtio_video_queue_detach_resources_resp {
        le32 result; /* VIRTIO_VIDEO_RESULT_* */
};
\end{lstlisting}
\begin{description}
\item[\field{result}] MUST be set to one of the variants of \field{enum
    virtio_video_result}.
\end{description}
\end{description}

\paragraph{Per resource commands}

\begin{description}
\item[VIRTIO_VIDEO_CMD_RESOURCE_ATTACH] Attach memory entries to use as a video
  buffer.

The driver sends this command with \field{struct virtio_video_resource_attach}.
\begin{lstlisting}
struct virtio_video_resource_attach {
        le32 cmd_type;
        le32 stream_id;
        le32 queue_type;
        le32 resource_id;
        union virtio_video_resource resources[];
};

struct virtio_video_resource_attach_resp {
        le32 result; /* VIRTIO_VIDEO_RESULT_* */
};
\end{lstlisting}
\begin{description}
\item[\field{cmd_type}] MUST be set to VIRTIO_VIDEO_CMD_RESOURCE_ATTACH.
\item[\field{stream_id}] is the ID of the stream.
\item[\field{queue_type}] is the direction of the queue.
\item[\field{resource_id}] is the ID of the resource. If the \field{queue_type}
  indicates the queue for an image format, \field{resource_id} MUST be an
  integer less than \field{cur_image_buffers} in \field{virtio_video_params}
  obtained via VIRTIO_VIDEO_GET_PARAMS. Otherwise, \field{resource_id} MUST be
  less than \field{cur_bitstream_buffers}.
\item[\field{resources}] specifies memory regions that will be attached. Its
  length of the array depends on the value of \field{planes_layout} in
  \field{virtio_video_params} obtained via VIRTIO_VIDEO_GET_PARAMS. If it is
  VIRTIO_VIDEO_PLANES_LAYOUT_SINGLE_BUFFER, the length MUST be 1. If it is
  VIRTIO_VIDEO_LAYOUT_PER_PLANE, the length MUST be equal to \field{num_planes}
  in \field{virtio_video_params}.

  The struct \field{virtio_video_resource} is defined as follows:
  \begin{lstlisting}
union virtio_video_resource {
        struct virtio_video_resource_sg_list sg_list;
        struct virtio_video_resource_object object;
};
\end{lstlisting}
\begin{description}
  \item[sg_list] represents a scatter-gather list. This field is valid when
    VIRTIO_VIDEO_MEM_TYPE_GUEST_PAGES is set for the specified queue type in
    VIRTIO_VIDEO_CMD_STREAM_CREATE command. The struct
    \field{virtio_video_resource_sg_list} is defined as follows:
    \begin{lstlisting}
struct virtio_video_resource_sg_entry {
        le64 addr;
        le32 length;
        u8 padding[4];
};

struct virtio_video_resource_sg_list {
        le32 num_entries;
        u8 padding[4];
        struct virtio_video_resource_sg_entry entries[];
};
\end{lstlisting}
The \field{num_entries} in \field{virtio_video_resource_sg_list} is the number
of \field{virtio_video_resource_sg_entry} instances that follow.

Each field in \field{virtio_video_resource_sg_entry} is used as follows:
    \begin{description}
    \item[\field{addr}] is the physical guest address.
    \item[\field{length}] is the length of the resource.
    \end{description}
  \end{description}
  \item[object] represents an object exported from other virtio devices as
    defined in \ref{sec:Basic Facilities of a Virtio Device / Exporting
      Objects}. This field is valid when VIRTIO_VIDEO_MEM_TYPE_VIRTIO_OBJECT is
    set for the specified queue type in VIRTIO_VIDEO_CMD_STREAM_CREATE command.
    The struct \field{virtio_video_resource_object} is defined as follows:
    \begin{lstlisting}
struct virtio_video_resource_object {
        u8 uuid[16];
};
    \end{lstlisting}
    \begin{description}
    \item[uuid] is a version 4 UUID specified by
      \hyperref[intro:rfc4122]{[RFC4122]}.
    \end{description}
\end{description}

The device responds with \field{struct virtio_video_resource_attach_resp}:
\begin{lstlisting}
struct virtio_video_resource_attach_resp {
        le32 result; /* VIRTIO_VIDEO_RESULT_* */
};
\end{lstlisting}
\begin{description}
\item[\field{result}] MUST be set to one of the variants of \field{enum
    virtio_video_result}.
\end{description}

\item[VIRTIO_VIDEO_CMD_RESOURCE_QUEUE] Add a buffer to the device's
queue.

\begin{lstlisting}
enum virtio_video_enqueue_flag {
        VIRTIO_VIDEO_ENQUEUE_FLAG_FORCE_KEY_FRAME = 0,
};

struct virtio_video_resource_queue {
        le32 cmd_type;
        le32 stream_id;
        le32 queue_type; /* One of VIRTIO_VIDEO_QUEUE_TYPE_* types */
        le32 resource_id;
        le32 flags; /* Bitmask with VIRTIO_VIDEO_ENQUEUE_FLAG_* */
        u8 padding[4];
        le64 timestamp;
        le32 data_sizes[VIRTIO_VIDEO_MAX_PLANES];
};
\end{lstlisting}
\begin{description}
\item[\field{cmd_type}] MUST be set to VIRTIO_VIDEO_CMD_RESOURCE_QUEUE.
\item[\field{stream_id}] is the ID for the stream the resource belongs to.
\item[\field{queue_type}] direction of the queue this resource belongs to.
\item[\field{resource_id}] is the ID of the resource to be queued.
\item[\field{flags}] is a bitmask of VIRTIO_VIDEO_ENQUEUE_FLAG_* values
  representing requirements when processing the resource. If the driver doesn't
  have any requirement, it MUST set this value to 0.
  \begin{description}
    \item[\field{VIRTIO_VIDEO_ENQUEUE_FLAG_FORCE_KEY_FRAME}] The frame MUST be
      encoded as a key frame. (only for encoders)
  \end{description}
\item[\field{timestamp}] is an abstract sequence counter that can be
  used for synchronization. When \field{queue_type} is set to
  VIRTIO_VIDEO_QUEUE_TYPE_INPUT, the driver MUST set this field to a unique
  value per frame. If using multiple VIRTIO_VIDEO_RESOURCE_QUEUE requests per
  frame, then the timestamps for a given frame MUST be identical.
  For VIRTIO_VIDEO_QUEUE_TYPE_OUTPUT, the driver MUST set it to 0.
\item[\field{data_sizes}] number of data bytes used for each plane. The driver
  MUST set this for each plane of an input buffer. For output buffers, the
  driver MUST set this to zero.
\end{description}

The device responds with \field{virtio_video_resource_queue_resp}:
\begin{lstlisting}
enum virtio_video_dequeue_flag {
        VIRTIO_VIDEO_DEQUEUE_FLAG_ERR = 0,
        VIRTIO_VIDEO_DEQUEUE_FLAG_EOS,

        /* Encoder only */
        VIRTIO_VIDEO_DEQUEUE_FLAG_KEY_FRAME,
        VIRTIO_VIDEO_DEQUEUE_FLAG_P_FRAME,
        VIRTIO_VIDEO_DEQUEUE_FLAG_B_FRAME,
};

struct virtio_video_resource_queue_resp {
        le32 result;
        le32 flags;
        le64 timestamp;
        le32 data_sizes[VIRTIO_VIDEO_MAX_PLANES];
};
\end{lstlisting}
\begin{description}
\item[\field{result}] MUST be set to one of variants of \field{enum
    virtio_video_result}.
\item[\field{flags}] is a bitmask of VIRTIO_VIDEO_DEQUEUE_FLAG_* flags.
  \begin{description}
  \item[\field{VIRTIO_VIDEO_DEQUEUE_FLAG_ERR}] When this flag is set, the data
    might have been corrupted or the process was terminated by
    VIRTIO_VIDEO_CMD_QUEUE_CLEAR.
  \item[\field{VIRTIO_VIDEO_DEQUEUE_FLAG_EOS}] When this flag is set, this is
    the last frame for the current stream.
  \item[\field{VIRTIO_VIDEO_DEQUEUE_FLAG_KEY_FRAME}] When this flag is set, the
    buffer contains a compressed image which is a key frame. (only for encoders)
  \item[\field{VIRTIO_VIDEO_DEQUEUE_FLAG_P_FRAME}] When this flag is set, the
    buffer contains contain only differences to a previous key frame. (only for
    encoders)
  \item[\field{VIRTIO_VIDEO_DEQUEUE_FLAG_B_FRAME}] When this flag is set,
    the buffer contains only the differences between the current frame and both
    the preceding and following key frames to specify its content. (only for
    encoders)
  \end{description}
\item[\field{timestamp}] is an abstract sequence counter that can be
  used for synchronization. For an output buffer, the device MUST copy
  the \field{timestamp} of the input buffer this output buffer was
  produced from.
\item[\field{timestamp}] is an abstract sequence counter that can be
  used for synchronization. When \field{queue_type} is set to
  VIRTIO_VIDEO_QUEUE_TYPE_INPUT, the device MUST set this field to 0.
  For VIRTIO_VIDEO_QUEUE_TYPE_OUTPUT, the device MUST copy the \field{timestamp}
  from requests of input buffers the frame of this response was produced from.
\item[\field{data_sizes}] is the size written by the device, for each valid
  plane.
\end{description}

\begin{itemize*}
\item For each VIRTIO_VIDEO_CMD_RESOURCE_QUEUE request, the device MUST send a
  response to the queue request with VIRTIO_VIDEO_OK_NODATA when it has finished
  processing the buffer successfully.
\item The device MUST mark a buffer that triggered a processing error with the
  VIRTIO_VIDEO_BUFFER_F_ERR flag.
\item The device MUST mark the last buffer with the VIRTIO_VIDEO_BUFFER_F_EOS
  flag to denote completion of the drain sequence.
\item In case of encoder, to denote a particular frame type the device MUST mark
  the respective buffer with VIRTIO_VIDEO_BUFFER_IFRAME,
  VIRTIO_VIDEO_BUFFER_PFRAME, or VIRTIO_VIDEO_BUFFER_BFRAME.
\item If the processing was stopped due to VIRTIO_VIDEO_CMD_QUEUE_CLEAR, the
  device MUST set \field{result} to VIRTIO_VIDEO_RESP_OK_NODATA and set the bit
  of VIRTIO_VIDEO_BUFFER_FLAG_ERR in \field{flags} in
  \field{virtio_video_resource_queue_resp}.
\item The driver and device MUST follow requirements about buffer ownership
  explained in \ref{sec:Device Types / Video Device / Device Operation / Buffer
    lifecycle}.
\end{itemize*}
\end{description}

\subsubsection{Event Virtqueue}

While processing buffers, the device can send asynchronous event notifications
to the driver. The behaviour depends on the exact stream. For example, the
decoder device sends a resolution change event when it encounters new resolution
metadata in the stream.

The device reports events on the event queue. The driver initially populates the
queue with device-writeable buffers. When the device needs to report an event,
it fills a buffer and notifies the driver. The driver consumes the event and
adds a new buffer to the virtqueue.
\begin{lstlisting}
enum virtio_video_event_type {
        /* For all devices */
        VIRTIO_VIDEO_EVENT_ERROR = 0x0100,

        /* For decoder only */
        VIRTIO_VIDEO_EVENT_DECODER_RESOLUTION_CHANGED = 0x0200,
};

struct virtio_video_event {
        le32 event_type; /* One of VIRTIO_VIDEO_EVENT_* types */
        le32 stream_id;
};
\end{lstlisting}

\begin{description}
\item[\field{event_type}] type of the triggered event.
\item[\field{stream_id}] id of the source stream.
\end{description}

The device MUST send VIRTIO_VIDEO_EVENT_DECODER_RESOLUTION_CHANGED whenever it
encounters new resolution data in the stream. This includes the case of the
initial device configuration after metadata has been parsed and the case of
dynamic resolution change.

\subsubsection{Buffer life cycle}
\label{sec:Device Types / Video Device / Device Operation / Buffer
  life cycle}

The state machine in Figure~\ref{fig:buffer-lifecycle} shows the life
cycle of a video buffer.
``ATTACH'' and ``QUEUE'' on edges represent per-resource device operations
VIRTIO_VIDEO_CMD_RESOURCE_ATTACH and VIRTIO_VIDEO_CMD_RESOURCE_QUEUE,
respectively.
The edges labeled with ''detach'' represents the following cases:
\begin{itemize}
\item VIRTIO_VIDEO_CMD_STREAM_DESTROY is completed,
\item VIRTIO_VIDEO_CMD_QUEUE_DETACH_RESOURCES is completed, or
\item another resource is attached for the resource_id by
  VIRTIO_VIDEO_CMD_RESOURCE_ATTACH.
\end{itemize}

VIRTIO_VIDEO_CMD_STREAM_DESTROY or VIRTIO_VIDEO_CMD_QUEUE_DETACH_RESOURCES.

\begin{figure}[h]
  \centering
  \includegraphics[width=\textwidth]{images/generated/video-buffer-lifecycle.png}
  \caption{Life Cycle of a Buffer}
  \label{fig:buffer-lifecycle}
\end{figure}


\drivernormative{\subparagraph}{Buffer life cycle}{Device Types / Video Device /
  Device Operation / Buffer life cycle}

The following table shows whether the driver can read or write each buffer in
each state in Figure~\ref{fig:buffer-lifecycle}. The driver MUST not read or
write buffers in the state that doesn't permit.
\begin{center}
  \begin{tabular}{|c|c|c|}
    \hline
    State & Input buffers & Output buffers \\
    \hline
    Detached & Read / Write & Read \\
    Queued   & -            & -    \\
    Dequeued & Read / Write & Read \\
    \hline
  \end{tabular}
\end{center}

\devicenormative{\subparagraph}{Buffer life cycle}{Device Types / Video
  Device / Device Operation / Buffer life cycle}

The following table shows whether the device can read or write each buffer in
each state in Figure~\ref{fig:buffer-lifecycle}. The device MUST not read or
write buffers in the state that doesn't permit.
\begin{center}
  \begin{tabular}{ |c|c|c| }
    \hline
    State & Input buffers & Output buffers \\
    \hline
    Detached  & -    & - \\
    Queued   & Read & Read / Write \\
    Dequeued & -    & Read \\
    \hline
  \end{tabular}
\end{center}
